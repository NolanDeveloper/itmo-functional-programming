\documentclass[fontsize=14pt, paper=a4, pagesize, DIV=calc]{article}

\usepackage[T2A]{fontenc}
\usepackage[utf8]{inputenc}
\usepackage[russian]{babel}
\usepackage{indentfirst} % Красная строка

\newcommand{\lam}[2]{\backslash #1 \rightarrow #2}

\begin{document}

\section{Домашнее задание(Денис М.)}

\begin{enumerate}
\setcounter{enumi}{-1}
\item Покажите, что

\begin{enumerate}
\item {\bf I = S K K}

$\begin{array}{lcl} 
S K K & = & (\lam{fgx}{f x (g x)}) K K \\ 
      & =_{\beta} & (\lam{gx}{K x (g x))} K \\
      & =_{\beta} & \lam{x}{K x (K x)} \\
      & = & \lam{x}{x} \\
      & = & I
\end{array}$

\item {\bf K* = K I}

$\begin{array}{lcl} 
K I & = & (\lam{xy}{x}) I \\ 
    & =_{\beta} & \lam{y}{I} \\ 
    & = & \lam{yx}{x} \\ 
    & =_{\alpha} & \lam{zx}{x} \\ 
    & =_{\alpha} & \lam{zy}{y} \\ 
    & =_{\alpha} & \lam{xy}{y} \\ 
    & = & K*
\end{array}$

\end{enumerate}

\item Выделите свободные и связанные переменные в термах и осуществите подстановки:

\begin{enumerate}
\item $x(\lam{xy}{y (x w) u})y[x := \lam{z}{z}]$

Свободные: w, u и только снаружи лямбды x, y

Связанные: x и y внутри лямбды 

$((x(\lam{xy}{y (x w) u})y)[x := \lam{z}{z}] = ((\lam{z}{z})(\lam{xy}{y (x w) u})y)$

\item $(\lam{x}{x (\lam{y}{y x}) w}) (\lam{x}{v}) [w := y (\lam{v}{v x})] $

Свободные: w, v

Связанные: все вхождения x и y

$(\lam{x}{x (\lam{y}{y x}) w}) (\lam{x}{v}) [w := y (\lam{v}{v x})] \\
= (\lam{z}{z (\lam{y}{y z}) (y (\lam{v}{v x}))}) (\lam{x}{v})$

\end{enumerate}

\item Уберите лишние скобки и осуществите бета-преобразование(если это возможно):

\begin{enumerate}

\item $((\lam{x}{(\lam{y}{((x y) z)})}) (a (b c))) \\
= (\lam{xy}{x y z}) (a (b c)) \\
=_{\beta} \lam{y}{a (b c) y z}  $

\item $(((m n) m) (\lam{x}{((x (u v)) y)})) \\
= m n m (\lam{x}{x (u v) y}) $

\item $((\lam{x}{(\lam{y}{((y x) x)})}) (x (x y)) y) \\
= (\lam{xy}{y x x}) (x (x y)) y \\
=_{\alpha} (\lam{xz}{z x x}) (x (x y)) y \\
=_{\beta} (\lam{z}{z (x (x y)) (x (x y))}) y \\
=_{\beta} y (x (x y)) (x (x y))
$

\end{enumerate}

\item Покажите, что

{\bf B = S (K S) K }

$\begin{array}{lcl} 
S (K S) K & = & (\lam{fgx}{f x (g x)}) (K S) K \\ 
          & =_{\beta} & (\lam{gx}{K S x (g x)}) K \\ 
          & =_{\beta} & \lam{x}{K S x (K x)} \\ 
          & = & \lam{x}{(\lam{xy}{x}) S x (K x)} \\ 
          & =_{\beta} & \lam{x}{(\lam{y}{S}) x (K x)} \\ 
          & =_{\beta} & \lam{x}{S (K x)} \\ 
          & = & \lam{x}{(\lam{f g x}{f x(g x)}) (K x)} \\ 
          & =_{\alpha} & \lam{x}{(\lam{f g y}{f y(g y)}) (K x)} \\ 
          & =_{\beta} & \lam{xgy}{K x y (g y)} \\ 
          & = & \lam{xgy}{(\lam{xy}{x}) x y (g y)} \\ 
          & =_{\beta} & \lam{xgy}{(\lam{y}{x}) y (g y)} \\ 
          & =_{\beta} & \lam{xgy}{x (g y)} \\ 
          & =_{\alpha} & \lam{fgy}{f (g y)} \\ 
          & =_{\alpha} & \lam{fgx}{f (g x)} \\ 
          & = & B
\end{array}$

\item Напишите терм, ведущий себя как логическая связка $xor$

$xor = \lam{b1}{\lam{b2}{b1\ (b2\ fls\ tru)\ b2}}$

\item Напишите терм, возводящий число Чёрча в степень(b – это основание, 
а e – это показатель степени):

Неправильно: $pow = \lam{be}{e (mult\ b) b}$ 

Понял ошибку. Начинать нужно с единицы, а не с {\it b}

Правильно будет: $pow = \lam{be}{e (mult\ b) \lam{sz}{sz}}$

\item Напишите терм, вычитающий единицу из числа Чёрча:

$pred = \lam{n}{p1(n (\lam{t}{pair (p2\ t) (succ (p1\ t))}) (pair\ 0\ 0)} $

\end{enumerate}

\end{document}
