\documentclass[12pt, a4paper] {article} 
\usepackage[utf8] {inputenc} 
\usepackage[T2A]{fontenc} 
\usepackage[english, russian] {babel} 
\usepackage[usenames,dvipsnames]{xcolor} 
\usepackage{listings,longtable,amsmath,amsfonts,graphicx,tikz,amssymb} 
\usepackage{indentfirst,verbatim} 
\usepackage[top=30pt,bottom=30pt,left=48pt,right=46pt]{geometry} 
\usepackage{proof}
\AtBeginDocument{\colorlet{defaultcolor}{.}}

\lstnewenvironment{code}{\lstset{language=Haskell,basicstyle=\small}}{}

\begin{document}
\frenchspacing
\pagestyle{empty}

\section{Конспект практики}

\subsection{Типы}

На лямбда-термы можно навесить \emph{типы}. Определение типов не даём, оно есть
в лекции, да все и так примерно его понимают.

Типизируются только \emph{сильно нормализуемые} термы, то есть те термы, которые
можно привести к нормальной форме независимо от последовательности редукций.

Чтобы задать правила типизации, введём понятие \emph{контекста}. Контекстом
назовём некоторое соответствие между именами переменных и связанными с ними
типами.

К примеру, в языке Си:

\begin{lstlisting}[language=C,basicstyle=\small]
int absurd(int a, char b, char *i)
{
        int c = 5;
        for (int i = 0; i < 15; ++i) {
                i -= 1; // point 1
                long r = i;
                i *= 0; // point 2
        }
        c = a + (int)b; // point 3
}
\end{lstlisting}

В точке 1 контекст можно принять за $\{a : \mathrm{int},\, b : \mathrm{char},\,
c : \mathrm{int},\, i : \mathrm{int}\}$, в точке 2~--- за $\{a : \mathrm{int},\,
b : \mathrm{char},\, c : \mathrm{int},\,  i : \mathrm{int},\, r :
\mathrm{long}\}$, в точке 3~--- за $\{a : \mathrm{int},\, b : \mathrm{char},\, c
: \mathrm{int},\, i : \mathrm{char *}\}$\footnote{Конечно, всё сложнее: в
контекст также входит, к примеру, сама функция absurd, да и вообще все видимые
функции и глобальные переменные, но мы тут не Си изучаем.}.
Контекст нужен для того, чтобы анализировать небольшие выражения~--- к примеру,
``\texttt{i -= 1}'',~--- помня про окружение (\emph{контекст}) данного
высказывания лишь необходимое.

Для типизации используются три правила:

\begin{enumerate}
  \item $$(x: \sigma) \in \Gamma \Longrightarrow \Gamma \vdash x: \sigma$$
  \item $$\infer{\Gamma \vdash A B : \tau}
                {\Gamma \vdash A : \sigma \to \tau & \Gamma \vdash B : \sigma}$$
  \item $$\infer{\Gamma \vdash \lambda x^{\color{red} \sigma}.~M : \sigma \to \tau}
                {\{x : \sigma\} \cup \Gamma \vdash M : \tau}$$
\end{enumerate}

\subsection{Типизация \textbf{B}}

Терм замкнутый, так что берём пустой контекст.

\begin{enumerate}
  \item $\lambda f~g~x.~f~(g~x)$~--- это абстракция. Применяем правило 3.
    $${\color{red} \infer[3]{\emptyset \vdash \lambda f~g~x.~f~(g~x) : ?_1 \to ?_2}
            {\{f: ?_1\} \vdash \lambda g~x.~f~(g~x) : ?_2} }$$
  \item $\lambda g~x.~f~(g~x)$~--- это абстракция. Применяем правило 3. Заметим,
    что $?_2$~--- это тип-стрелка, и разобьём его: $?_2 = ?_3 \to ?_4$.
    $$\infer[3]{\emptyset \vdash \lambda f~g~x.~f~(g~x) : ?_1 \to {\color{red} ?_3 \to ?_4}}{
      \color{red}
      \infer[3]{\color{defaultcolor} \{f: ?_1\} \vdash \lambda g~x.~f~(g~x) : {\color{red} ?_3 \to ?_4}}{
               {\{f: ?_1, g: ?_3\} \vdash \lambda x.~f~(g~x) : ?_4 }
      }}$$
  \item $\lambda x.~f~(g~x)$~--- это снова абстракция (но последняя). Снова
    правило 3. $?_4$~--- это тип-стрелка, пусть $?_4 = ?_5 \to ?_6$.
    $$\infer[3]{\emptyset \vdash \lambda f~g~x.~f~(g~x) : ?_1 \to ?_3 \to {\color{red}?_5 \to ?_6}}{
      \infer[3]{\{f: ?_1\} \vdash \lambda g~x.~f~(g~x) : ?_3 \to {\color{red}?_5 \to ?_6}}{
      \color{red} \infer[3]{\color{defaultcolor} \{f: ?_1, g: ?_3\} \vdash \lambda x.~f~(g~x) : {\color{red}?_5 \to ?_6} }{
               {\{f: ?_1, g: ?_3, x: ?_5\} \vdash f~(g~x) : ?_6 }
      }}}$$
  \item $f~(g~x)$~--- это, наконец, аппликация. Применимо правило 2.
    $$\infer[3]{\emptyset \vdash \lambda f~g~x.~f~(g~x) : ?_1 \to ?_3 \to ?_5 \to ?_6}{
      \infer[3]{\{f: ?_1\} \vdash \lambda g~x.~f~(g~x) : ?_3 \to ?_5 \to ?_6}{
      \infer[3]{\{f: ?_1, g: ?_3\} \vdash \lambda x.~f~(g~x) : ?_5 \to ?_6 }{
      \color{red} \infer[2]{\color{defaultcolor} \{f: ?_1, g: ?_3, x: ?_5\} \vdash f~(g~x) : ?_6 }{
               \{f: ?_1, g: ?_3, x: ?_5\} \vdash f : ?_7 \to ?_6 & \{f: ?_1, g: ?_3, x: ?_5\} \vdash g x : ?_7
      }}}}$$
  \item Пойдём в левую ветку. Там применимо правило типизации 1. Значит,
    $?_1 = ?_7 \to ?_6$.
    $$\infer[3]{\emptyset \vdash \lambda f~g~x.~f~(g~x) : {\color{red} (?_7 \to ?_6)} \to ?_3 \to ?_5 \to ?_6}{
      \infer[3]{\{f: {\color{red} ?_7 \to ?_6}\} \vdash \lambda g~x.~f~(g~x) : ?_3 \to ?_5 \to ?_6}{
      \infer[3]{\{f: {\color{red} ?_7 \to ?_6}, g: ?_3\} \vdash \lambda x.~f~(g~x) : ?_5 \to ?_6 }{
      \infer[2]{\{f: {\color{red} ?_7 \to ?_6}, g: ?_3, x: ?_5\} \vdash f~(g~x) }{
               {\color{red} \infer[1]{\color{defaultcolor} \{f: {\color{red} ?_7 \to ?_6}, g: ?_3, x: ?_5\} \vdash f : {\color{red} ?_7 \to ?_6}}{}} &
                \{f: {\color{red} ?_7 \to ?_6}, g: ?_3, x: ?_5\} \vdash g x : ?_7
      }}}}$$
  \item Теперь пойдём в правую ветку. Там аппликация и правило типизации 2. {\scriptsize
    $$\infer[3]{\emptyset \vdash \lambda f~g~x.~f~(g~x) : (?_7 \to ?_6) \to ?_3 \to ?_5 \to ?_6}{
      \infer[3]{\{f: ?_7 \to ?_6\} \vdash \lambda g~x.~f~(g~x) : ?_3 \to ?_5 \to ?_6}{
      \infer[3]{\{f: ?_7 \to ?_6, g: ?_3\} \vdash \lambda x.~f~(g~x) : ?_5 \to ?_6 }{
      \infer[2]{\{f: ?_7 \to ?_6, g: ?_3, x: ?_5\} \vdash f~(g~x) }{
               \infer[1]{\{f: ?_7 \to ?_6, g: ?_3, x: ?_5\} \vdash f : ?_7 \to ?_6}{} &
      {\color{red} \infer[2]{\color{defaultcolor} \{f: ?_7 \to ?_6, g: ?_3, x: ?_5\} \vdash g x : ?_7}{
               \{f: ?_7 \to ?_6, g: ?_3, x: ?_5\} \vdash g : ?_8 \to ?_7 &
               \{f: ?_7 \to ?_6, g: ?_3, x: ?_5\} \vdash x : ?_8
      }}}}}}$$ }
  \item Пойдём в левую ветку. Там применимо правило 1, тогда $?_3 = ?_8 \to ?_7$.{\scriptsize
    $$\infer[3]{\emptyset \vdash \lambda f~g~x.~f~(g~x) : (?_7 \to ?_6) \to ({\color{red} ?_8 \to ?_7}) \to ?_5 \to ?_6}{
      \infer[3]{\{f: ?_7 \to ?_6\} \vdash \lambda g~x.~f~(g~x) : ({\color{red} ?_8 \to ?_7}) \to ?_5 \to ?_6}{
      \infer[3]{\{f: ?_7 \to ?_6, g: {\color{red} ?_8 \to ?_7}\} \vdash \lambda x.~f~(g~x) : ?_5 \to ?_6 }{
      \infer[2]{\{f: ?_7 \to ?_6, g: {\color{red} ?_8 \to ?_7}, x: ?_5\} \vdash f~(g~x) }{
               \infer[1]{\{f: ?_7 \to ?_6, g: {\color{red} ?_8 \to ?_7}, x: ?_5\} \vdash f : ?_7 \to ?_6}{} &
      \infer[2]{\{f: ?_7 \to ?_6, g: {\color{red} ?_8 \to ?_7}, x: ?_5\} \vdash g x : ?_7}{
               {\color{red} \infer[1]{\color{defaultcolor} \{f: ?_7 \to ?_6, g: {\color{red} ?_8 \to ?_7}, x: ?_5\} \vdash g : {\color{red} ?_8 \to ?_7}}{}} &
               \{f: ?_7 \to ?_6, g: {\color{red} ?_8 \to ?_7}, x: ?_5\} \vdash x : ?_8
      }}}}}$$}
  \item Пойдём в правую ветку и получим итоговое дерево путём применения
    правила 1 и замечания, что тогда $?_8 = ?_5$.{\scriptsize
    $$\infer[3]{\emptyset \vdash \lambda f~g~x.~f~(g~x) : (?_7 \to ?_6) \to ({\color{red} ?_5} \to ?_7) \to ?_5 \to ?_6}{
      \infer[3]{\{f: ?_7 \to ?_6\} \vdash \lambda g~x.~f~(g~x) : ({\color{red} ?_5} \to ?_7) \to ?_5 \to ?_6}{
      \infer[3]{\{f: ?_7 \to ?_6, g: {\color{red}?_5} \to ?_7\} \vdash \lambda x.~f~(g~x) : ?_5 \to ?_6 }{
      \infer[2]{\{f: ?_7 \to ?_6, g: {\color{red}?_5} \to ?_7, x: ?_5\} \vdash f~(g~x) }{
               \infer[1]{\{f: ?_7 \to ?_6, g: {\color{red}?_5} \to ?_7, x: ?_5\} \vdash f : ?_7 \to ?_6}{} &
      \infer[2]{\{f: ?_7 \to ?_6, g: {\color{red}?_5} \to ?_7, x: ?_5\} \vdash g x : ?_7}{
               \infer[1]{\{f: ?_7 \to ?_6, g: {\color{red}?_5} \to ?_7, x: ?_5\} \vdash g : {\color{red}?_5} \to ?_7}{} &
               {\color{red} \infer[1]{\color{defaultcolor} \{f: ?_7 \to ?_6, g: {\color{red}?_5} \to ?_7, x: ?_5\} \vdash x : {\color{red} ?_5}}{}}
      }}}}}$$}
\end{enumerate}

Полученный тип~--- $(?_7 \to ?_6) \to (?_5 \to ?_7) \to ?_5 \to ?_6$~---
совпадает с интуицией.

\section{Домашнее задание}

Найдите наиболее общий тип:

\begin{enumerate}
  \item $\lambda x.~x$;
  \item $\lambda x~x.~x$;
  \item $\lambda x^{\alpha \to \alpha}.~x$;
  \item $\lambda x.~x~\mathbf{K}~\mathbf{I}~x$;
  \item $\lambda x~y~z.~x~(y~z~(z~x))$.
\end{enumerate}

Населите как можно большим количеством замкнутых термов типы или покажите, как
сконструировать бесконечное их число:

\begin{enumerate}
\setcounter{enumi}{5}
  \item $\alpha \to \alpha$;
  \item $\alpha \to \alpha \to \alpha$;
  \item $\alpha \to \alpha \to \alpha \to \alpha$;
  \item $\alpha$;
  \item $\alpha \to \beta \to \alpha$;
  \item $\alpha \to \beta \to (\alpha \to \beta \to \gamma) \to \gamma$;
  \item $\alpha \to (\alpha \to \alpha) \to \alpha$;
  \item $((\alpha \to \beta) \to \alpha) \to (\alpha \to \alpha \to \beta) \to \beta$;
  \item $((((\alpha \to \beta) \to \alpha) \to \alpha) \to \beta) \to \beta$.
\end{enumerate}

Типизируйте по Чёрчу:

\begin{enumerate}
\setcounter{enumi}{14}
  \item $\mathbf{S}~\mathbf{K}~\mathbf{K}$;
  \item $\mathbf{S}~\mathbf{K}~\mathbf{I}$.
\end{enumerate}

Честно, аксиоматически, типизируйте (можно писать в одно дерево, не по шагам):

\begin{enumerate}
\setcounter{enumi}{16}
  \item Комбинатор $\mathbf{S}$.
\end{enumerate}

\end{document}
